\chapter{Conclusions}
\label{sec:chap6}
In this section we give a brief summary of our work, discuss our results and offer some ideas for further research in the area of 3D classification.
\section{Summary}
In this work we explored different artificial neural-network-based systems for classification of 3D models. We performed a broad survey and gave a brief description of the individual systems. We also used the code published by the authors of the original publications and used it to successfully train and test several of these networks. In order to achieve this, we constructed a reasonably independent framework which enables to run and compare networks implemented in different machine learning frameworks. The conversions of 3D meshes to point clouds, voxel grids and multi-view images are also part of our framework. We have tested all the networks and data conversion on ModelNet40 and ShapeNetCore datasets so our framework is ready to be used on these and we believe it can be extended on similarly structured datasets without much difficulties. \par
As for our results, we generally failed to replicate the original results on ModelNet40. All the networks performed reasonably well, but we have achieved a couple percent less than the reported accuracy. This can be caused by the sensitivity of neural networks to hyperparameter tuning. We suppose that the authors of the original papers spent a considerable amount of time on  making their networks achieve the best results possible. In general they do not describe the exact training setup in their publications and we did not have enough resources to perform an exhaustive hyperparameter search for all the networks.
The relative performance of the networks was confirmed by our results; networks with higher reported accuracy generally performed better in our experiments as well. The differences are more significant with ShapeNetCore, multi-view networks outperforming the point-cloud-based networks. \par
 The networks also differ greatly in the required training time, from weeks to several hours, and this can be a very important fact to consider when choosing an approach in practice. The voxel-based neural networks are very slow, multi-view-based networks are considerably faster and point-cloud-based and octree-based networks are the fastest. 

\section{Future Work}
The research in the field of artificial intelligence in general and in the area of neural networks in particular is progressing very quickly. As we were working on this thesis there already appeared some new publications (\cite{yu_multi-view_2018, you_pvnet:_2018, feng_gvcnn:_2018}) improving multi-view approaches to 3D classification which seem to be the most promising approach. It would be worth including these new systems to our framework. \par
Almost all of the networks we have tested describe in the papers and implement in the code a version for part segmentation of the 3D models. This is a problem of dividing a 3D model to logical parts such as dividing the model of a table to the top desk and the legs. It would be worth comparing all the different networks as well.\par
We hope that this work will lead to some practical improvements in the fields such as interior design, where artificial intelligence can solve menial tasks currently solved by experts. 
