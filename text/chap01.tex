\chapter{Introduction}
\label{sec:chap1}


\section{Motivation and Goals}
Recognition and generation of 3D shapes is quickly becoming one of the widely researched topic in the field of artificial intelligence. It can be applied in vast number of fields such as driving of autonomous cars, analysis of medical scans as well as various fields of computer graphics. We approach this problem from the standpoint of computer graphics as we are interested in developing tools for content creators, architects or interior designers. In order to develop such tools it is necessary to start with the simplest problem which is classification. 
There are many more or less successful approaches to 3D classification, most of them employing some kind of deep artificial neural network and their relative performance has never been objectively evaluated.\par
The goal of this thesis is therefore to test existing techniques for 3D classification and find out how difficult is to replicate their reported results. Also to compare them and evaluate their applicability in real world setting.
\section{Problem Statement, Scope of the Thesis}
As we intend to develop tools for computer graphics our input is in the form of a 3D mesh. 3D mesh is usually supplied as list of vertices - triplets of coordinates in euclidean space and list of faces - usually three vertices each a triangle. 3D mesh files can contain other information such as textures and materials, but we will be ignoring these. Our goal is to classify mesh files to several given categories using methods of supervised learning.\par
We can define the problem formally as follows: we are given set of training examples $\{(x_1,y_1),(x_2,y_2), ..., (x_n,y_n)\}$ where, in our case, $x_i$ is a 3D shape and $y_i$ is a numerical representation of corresponding label. Each shape belongs to exactly one class. The goal of classification can be formulated as learning a parametric model $P:X \rightarrow Y$ where $X$ is a space of possible 3D shapes and $Y$ is a space of labels. This model should be able to predict the correct class label for each model from $X$.
As the mesh format is not suitable for direct use with neural networks, we have to be able to convert meshes to other representations: point clouds, volumes, or images.
\par
We limit the scope of this thesis in the following ways. We will perform classification only on aligned 3D shapes, as no all networks can easily cope with arbitrary rotations. We will only consider simple classification, although most of the networks can be extended to perform part segmentation as well as new shape generation. We will not test any large ensembles of networks, although they produce better results, they are usually big and cumbersome to use, while achieving only marginal improvements. We also consider only networks with publicly available code as implementing all the different techniques is far beyond the scope of this work.

\section{Thesis Outline}
In this chapter \hyperref[sec:chap1]{1} we presented basics of the problem as well as our motivation for this work. In the second chapter \hyperref[sec:chap2]{2} we provide a brief introduction to artificial neural networks. In the third chapter \hyperref[sec:chap3]{3} we introduce different approaches to 3D classification and their implementations. In the fourth chapter \hyperref[sec:chap4]{4} we discuss chosen methods, describe datasets, procedures of data conversion and technical setup of our framework. In chapter five \hyperref[sec:chap5]{5} we present the setup and results of our experiments, comparison of tested networks and analysis of the dataset. In the final chapter \hyperref[sec:chap6]{6} we summarize our findings and suggest directions for future work.
